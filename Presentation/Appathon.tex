\documentclass[9pt]{beamer}
%
% Choose how your presentation looks.
%
% For more themes, color themes and font themes, see:
% http://deic.uab.es/~iblanes/beamer_gallery/index_by_theme.html
%
\mode<presentation>

{
  \usetheme{PaloAlto}      % or try Darmstadt, Madrid, Warsaw, ...
  \usecolortheme{whale} % or try albatross, beaver, crane, ...
  \usefonttheme{default}  % or try serif, structurebold, ...
  \setbeamertemplate{navigation symbols}{}
  \setbeamertemplate{caption}[numbered]
} 
\setbeamertemplate{navigation symbols}{}%remove navigation symbols at bottom

%I had problems compiling doc on X230 without next two lines
\usepackage{etex}
\reserveinserts{28}
%On X530 it worked without problems

%\usepackage[utf8x]{inputenc}
\input{Preamble.tex}
%\renewcommand{\baselinestretch}{1.2} %line spacing
%{\setstretch{1.0}\color{blue} text bla bla } for section strech
\renewcommand{\footnotesize}{\scriptsize}
%\usepackage[demo]{graphicx}
\usepackage{caption}
\usepackage{subcaption}

%for TikZ
\usepackage[mode=buildnew]{standalone}% requires -shell-escape


\title[Appathon]{Locust - Timing is crucial}
%\author{Dadoes}
\institute{Dadoes}
\date{25/01/2015}

\begin{document}

\begin{frame}
  \titlepage

\end{frame}

% Uncomment these lines for an automatically generated outline.
%\begin{frame}{Outline}
%  \tableofcontents
%\end{frame}

\section{Application Description}



\begin{frame}{Introduction}
	\begin{figure}
		\centering
		\includegraphics[width=0.7\textwidth]{pic/madagascar-locust-logo-DSCN9281-EN.jpg}
		\caption{Locust attack}
	\end{figure}
\end{frame}

	
\begin{frame}{Locust}
Main risk for food production in Africa and ME, also affecting Asia and Australia. Its timing of the attack not its severity.
\bigskip

\textbf{Case Study - Madagascar}

	\begin{itemize}
		\item 13 out of 22m people affected;
		\item 65\% of 41m hectares of agricultural land;
	 \end{itemize} 
	 Since 2013 Food and Agriculture organisation of UN is leading failing attemt to control thread:
	\begin{itemize}
		\item 1 025 hours flown for survey operations;
		\item 1 017 hours for control operations by three helicopters and one fixed-wing aircraft; 
		\item 28.4m freed 1.3m hectares and lead to pesticide over-spraying;
		\item This year spraying programme is underfunded by 10.6m; 
	 \end{itemize}
 
\end{frame}



\begin{frame}{Solution}
Using space data we want to indicate areas of hight probability of occurrence (starting location and spread direction).

Forming factors are:
\begin{itemize}
	\item eggs in sandy soils and near the sea;
	\item {heavy rainfall in Nov-Feb leads to overgrown;}
	\item {Drought conditions create overpopulation;}
 \end{itemize} 
Weather conditions affected by:
\begin{itemize}
	\item {sea temperature;}
	\item {cloud, wind;}
 \end{itemize} 
 
Relevant information can be obtained from LANDSAT and MODIS. For live data we aim to use SENTINEL 2.

\end{frame}

\begin{frame}{System overview}

	\begin{figure} %this is tikZ figure! dont include .tex
	%	\centering
		\includestandalone[height=.8\textheight]{pic/SoftwareWorkflow}

		%\caption{Locust attack}
	\end{figure}
\end{frame}


\begin{frame}{Software output - rainy and dry season (NDVI)}
	\begin{figure}
		\centering
		\begin{subfigure}{.5\textwidth}
			\centering
			\includegraphics[width=\textwidth]{pic/A2013345.png}
			\caption{Rainy season}
			\label{fig:sub1}
		\end{subfigure}%
		\begin{subfigure}{.5\textwidth}
			\centering
			\includegraphics[width=\textwidth]{pic/A2014153.png}
			\caption{Dry season}
			\label{fig:sub2}
		\end{subfigure}
%		\caption{Comparison}
%		\label{fig:test}
	\end{figure}

\end{frame}

\begin{frame}{Software output - difference}

	\begin{figure}
		\centering
		\includegraphics[width=0.7\textwidth]{pic/diff.png}
		%\caption{Locust attack}
	\end{figure}
\end{frame}


\end{document}
